\documentclass[12pt]{article}
\usepackage{amsthm}
\usepackage{graphicx}
\usepackage{pythonhighlight}
\usepackage[backend=bibtex]{biblatex}
\title{Sigma Protocols}
\usepackage{hyperref}
\author{Mmanu Chaturvedi (40291202)}
\newtheorem{definition}{Definition}
\addbibresource{references.bib}


\begin{document}
	\maketitle
	\tableofcontents
	
	
	\section{Introduction}
	This report attempts to understand some parts of sigma protocols.  Sigma protocols were proposed by Ronald Cramer in his thesis \cite{cramer1996modular}. The word `Sigma' was used because `Sig' can be used for zig-zag, symbolizing the three moves which constitute the sigma protocols and `ma' as a short for `Merlin-Arthur' games \cite{babai1988arthur}.  
	
	In order to define $\Sigma$-protocols, we need to define some prerequisites.  We define two entities, a prover $A$, and a verifier $B$.  $x$ is the common input to $A$ and $B$. $w$ is the private input to $A$.  The prover $A$ is trying to convince the verifier $B$ that it knows of the private input $w$.  And the aim is to do this as `quietly' as possible, i.e. the verifier must gain no knowledge other than whether $A$ knows $w$ or not.

	Mathematically, let $R$ be a binary relation, i.e.\ a subset of $\{0, 1\}^* \times \{0, 1\}^*$, and $(x, w) \in R $. We require that the number of bits in $w$ is polynomial in the number of bits in $x$, i.e. $\left|w\right| = p(\left|x\right|)$.  The protocol has the following form:
	
	\begin{enumerate}
		\item $A$ sends a message $a$ to $B$.
		\item $B$ sends a random $t$-bit string $c$.
		\item $A$ sends a reply $z$ that $B$ decides to accept or reject.
	\end{enumerate}
	
	\begin{figure}
	\includegraphics[scale=0.65]{sigma\_protocol.png}
	\caption{Sigma protocol, a conversation between A (prover) and B (verifier) is defined by $(x, a, z, r)$, where $r$ is the result (accept or reject).}
	\end{figure}
	
	The requirement is that $a$, $c$, $z$ and the decision($\phi$) are all calculated in polynomial time.  By polynomial time we mean that the computational complexity of calculating the values of $a$, $c$, $z$ and the decision is polynomial in the number of bits in the input to those calculations.
	
	\section{Completeness and Soundness}
	A proof system or a verifying procedure has two fundamental properties \cite{Goldreich_2001}, \emph{soundness} and \emph{completeness}.  The \emph{soundness} property makes sure
	of the ability of the verifier to protect itself from being convinced of a false statement, that is, if a statement is false, the verifier should reject it and if the statement is true, the verifier should definitely accept it. 
	
	The \emph{completeness} property, as opposed to the \emph{soundness} property focuses on the prover.  It ascertains that the prover has the ability to `convince' the verifier of true statements.
	
	It is important to remark the need to define these seemingly `obvious' definitions.  By \emph{Godel's Incompletness Theorem}, we know that not all statements can have a `proof system', so even if the statement is true there isn't a consistent proof system which can be known to prove it.
	
	\section{(Computational) Zero-Knowledge}
	
	In layman's terms, zero-knowledge proofs are proofs by which the verifier gains no `knowledge' other than whether the claimed statement is true or false.   While ignoring the definition of `knowledge' in itself, we define what is `gaining knowledge' and use that to define `zero-knowledge' proofs.  The idea of `zero-knowledge proofs' was proposed by Goldwasser et al.\ \cite{goldwasser2019knowledge}.
	
	To give a simple example, consider two people Alice and Bob, and imagine a large graph, $G$, is shared between the two.  Bob asks Alice questions about $G$ and Alice answers true or false.  Say Bob asks the question whether $G$ is Eulerian? Then clearly independent of what Alice says, Bob has \emph{gained no knowledge} because he could have answered the question efficiently (in polynomial time) by himself.  If however, Bob had asked if $G$ is Hamiltonian, and Alice would have said true or false, Bob would have \emph{gained knowledge} since he couldn't have calculated the answer to that question by himself (unless $\mathcal{P} = \mathcal{NP}$).  Thus Bob gains knowledge if after his interaction with Alice, he could compute something that he could not before the interaction.
	
	Mathematically, zero-knowledge proofs are defined in terms of interactive Turing machines.  They can be perfect zero-knowledge or computational zero-knowledge.  For the sake of this report, we skip the exact mathematical definition and refer the reader to the book \cite{Goldreich_2001}. Another level of approximation over the computational zero-knowledge leads to `honest-verifier zero-knowledge'.
	
	$(A, B)$ is called \emph{honest verifier zero-knowledge}, if a probabilistic polynomial-time Turing machine that when given input $x$ can produce accepting conversations, $(x, a, z, r)$, with the same distribution as between the prover, $A$ (who is given $(x, w) \in R$) and the verifier, $B$.  Note that we've skipped much of the mathematical definitions in this report, we need to define what's `same distribution', what is `probabilistic polynomial-time' Turing machine which is used for describing randomized algorithms, etc.  But we believe that the words capture the spirit of the mathematical idea.
	
	
	We define a $\Sigma$-protocol over a relation $R$, $\{0, 1\}^* \times \{0, 1\}^*$, as the following:
	
	
	
	\begin{definition}
		\cite{damgaard2002sigma} A protocol $\mathcal{P}$ is called a $\Sigma$-protocol for $R$ if:
		
		\begin{itemize}
			\item $\mathcal{P}$ is of the above 3-move form, and we have completeness: if $A, B$
			follow the protocol on input $x$ and private input $w$ to $B$ where $(x, w) \in
			R$, the verifier always accepts.
			\item For any $x$ and any pair of accepting conversations on input $x$,
			$(a, c, z)$, $(a, c' , z')$ where $c \ne c'$ , one can efficiently compute $w$ such
			that $(x, w) \in R$. This is sometimes called the special soundness property.
			\item There exists a polynomial-time simulator $M$, which on input $x$ and a
			random $c$ outputs an accepting conversation of the form $(a, c, z)$, with
			the same probability distribution as conversations between the honest
			$A, B$ on input $x$. This is sometimes called special honest-verifier zero-
			knowledge.
			
		\end{itemize}
		
	\end{definition}
	
	
%	\section{Definitions}
%	We start with a few definitions that would be useful in understanding sigma protocols.
%	
%	\begin{definition}	
%		An identification protocol is a triple $I = (G, P, V )$.
%		
%		\begin{itemize} 
%			\item $G$ is a probabilistic, key generation algorithm, that takes no input, and outputs a pair
%			$(vk , sk )$, where $vk$ is called the verification key and $sk$ is called the secret key.
%			\item $P$ is an interactive protocol algorithm called the prover, which takes as input a secret key $sk$, as output by $G$.
%			\item $V$ an interactive protocol algorithm called the verifier, which takes as input a verification key $vk$ , as output by $G$, and which outputs accept or reject.
%		\end{itemize}
%	\end{definition}
%	\begin{definition}
%		Attack Game for eavesdropping attack.
%		
%		For a given identification protocol $I = (G, P, V )$ and a given adversary $A$, the attack game runs as follows:
%		
%		\begin{itemize}
%			\item{Key generation phase}: The challenger runs $(vk , sk ) \leftarrow G()$, and sends $vk$ to A.
%			\item{Eavesdropping phase}: The adversary requests some number, say $Q$, of transcripts of conversations between $P$ and $V$. The challenger complies by running the interaction between $P$ and $V$ a total of $Q$ times, each time with $P$ initialized with input $sk$ and $V$ initialized with $vk$. The challenger sends these transcripts $T_1 ,\dots, T_Q$ to the adversary.
%			
%			\item{Impersonation attempt}: The challenger and $A$ interact, with the
%			challenger following the verifier’s algorithm $V$ (with input $vk$), and with $A$ playing the role
%			of a prover, but not necessarily following the prover’s algorithm $P$.
%			
%			We say that the adversary wins the game if $V$ outputs accept at the end of the interaction. We define $A$'s advantage with respect to $I$, denoted $ID2adv[A, I]$, as the probability that $A$ wins the
%			game. 
%		\end{itemize}
%	\end{definition}
%	
%	\begin{definition}
%		We say that an identification protocol $I$ is secure against eavesdropping
%		attacks if for all efficient adversaries $A$, the quantity $ID2adv[A, I]$ is negligible.
%	\end{definition}
	
	
	\section{Schnorr's identification protocol (Honest-verifier zero-knowledge)}
	
	In this section, we give an example of honest-verifier zero-knowledge protocol.  We first define the discrete log problem.
	
	\subsection{Discrete log problem for $Z_p^{*}$}
	
	Consider a group $Z_p^{*}$ on $\{1, \dots, p-1\}$ with the group operation as multiplication modulo $p$.  The discrete log problem is to find $w$ if given $h$, $g$ and $p$, where $h = g^w$ mod $p$.  Where $g$ is the generator of $Z_p^*$, i.e., $Z_p^* = \{g^1, g^2, \dots,g^{p-1}\}$. It is not known if the complexity class of discrete log problem is $\mathcal{P}$ or not \cite{odlyzko2000discrete}, but by experience we assume that this is a hard problem.
	
	\subsection{The protocol}
	
	
	
	
	The protocol of Schnorr's is an special case of Sigma protocol, so it consists of three steps.  We first define some prerequisites, let $p$ and $q$ be large prime numbers, such that $q | (p -1)$.  Such a prime $p$ is chosen so that \href{https://www.doc.ic.ac.uk/~mrh/330tutor/ch06s02.html}{DLP becomes difficult}.  Let $g$ be a generator for $Z_p^*$. And let the $h = g^w $ mod $p$, that is published by $A$. The following are the steps of Schnorr's sigma protocol:
	
	
	\begin{enumerate}
		\item$A$ chooses $r$ randomly in $Z_p$ and sends $a = g^r$ mod $p$ to $B$.
		\item $B$ chooses a $t$-bit $c$ randomly and sends it to $A$. Assume $2^t < p$.
		\item $A$ sends $z = r + cw$ mod $p$ to $B$, who checks that $g^z = ah^c$ mod $p$. The verifier, $B$, accepts if this is the case and rejects otherwise.
	\end{enumerate} 
	
	The transcript of the above would look like $(a, c, z)$. 
	
	\subsection{Concerete example}
	
	In this example we use $p = 20336$ and find a generator $g = 127$.
	\inputpythonfile{main.py}
	
	\section{Conclusion}
	
	In this report we document our understanding of the theory underlying sigma protocols in a non-mathematical manner.  Out of the many sigma protocols, we studied one protocol based on the hardness of the discrete log problem.  We found the book `Foundations of Cryptography' \cite{Goldreich_2001} a good resource for understanding the basics of zero-knowledge proofs (ZKP).  We also tried a short snippet in Python to understand how the sigma protocol based on Schnorr's identification scheme works.  Overall, this report just scratches the surface of the interesting field of ZKP.
	
	\printbibliography
\end{document}